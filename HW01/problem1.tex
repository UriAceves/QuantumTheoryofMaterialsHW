\begin{questions}
\question{
Radii
}
\begin{solution}
If the electron is moving in circles around a nucleus of charge $Ze$, with constant tangential velocity $v$, then in order to stay in any orbit, the centripetal force must be equal to the electric force generated by the electron and the nucleus. Mathematically we can express this as
\begin{equation}
  F = \underbrace{\frac{1}{4\pi\epsilon_0}\frac{Ze^2}{r^2}}_\text{Coulomb force} = \underbrace{\frac{m_ev^2}{r}}_\text{Centripetal force} .
  \label{centrip}
\end{equation}

For this problem I will assume that the angular momentum can take on values $\hbar$, $2\hbar$, $\ldots$, but never non-integer values. For circular orbits, the position vector of the electron $\vec{r}$ is always perpendicular to the linear momentum of the particle $\vec{p}$. The angular momentum $\vec{L}$ has magnitude $L=rp=m_evr$ in this case. As a consequence of this and the previously assumed values of the angular momentum we now know that,
\begin{equation}
  m_evr = n\hbar, \qquad n\in \mathbb{Z},
  \label{ang}
\end{equation}
if we isolate the velocity term from equation \ref{ang} then we obtain
\begin{equation}
  v = \frac{n\hbar}{m_er}.
  \label{vel}
\end{equation}

Now all we need to do is to insert eq. \ref{vel} into eq. \ref{centrip}
\begin{equation}
  \frac{m_ev^2}{r} = \frac{m_e}{r}\left(\frac{n\hbar}{m_er}\right)^2 = \frac{1}{4\pi\epsilon_0}\frac{Ze^2}{r^2},
  \label{eq:4}
\end{equation}
and finally we get
\begin{eqnarray*}
  \frac{\cancel{m_e}n^2\hbar^2}{r^{\cancel{3}}m_e^{\cancel{2}}} = \frac{Ze^2}{4\pi\epsilon_0\cancel{r^2}},\\
  \Rightarrow \frac{n^2\hbar^2}{rm_e} = \frac{Ze^2}{4\pi\epsilon_0},\quad
\end{eqnarray*}
\begin{equation}
r_n = \frac{4\pi\epsilon_0 \hbar^2}{m_ee^2} \frac{n^2}{Z} = \hlgreen{a_0\frac{n^2}{Z}},
\label{eq:5}
\end{equation}
where $a_0$ is the Bohr radius.
\end{solution}

\question{
Energies
}\begin{solution}
  From eq. \ref{centrip} we can obtain the kinetic energy $K$, using $r_n$
  \begin{eqnarray}
\frac{1}{4\pi\epsilon_0}\frac{Ze^2}{r_n^2} = \frac{m_ev^2}{r_n}
\Rightarrow m_ev^2 = \frac{1}{4\pi\epsilon_0}\frac{Ze^2}{r_n},\nonumber\\
\Rightarrow K = \frac{m_ev^2}{2} = \frac{1}{8\pi\epsilon_0}\frac{Ze^2}{r_n}.
\end{eqnarray}
On the other hand, the potential energy $U$ is easy to calculate, since it only comes from the Coulomb forces. $U$ is then
\begin{equation}
  U = - \frac{1}{4\pi\epsilon_0}\frac{Ze^2}{r_n}.
\end{equation}
At this point we use $E = K + U$, therefore
\begin{equation}
  E_n = \frac{1}{8\pi\epsilon_0}\frac{Ze^2}{r_n} - \frac{1}{4\pi\epsilon_0}\frac{Ze^2}{r_n} = - \frac{1}{8\pi\epsilon_0}\frac{Ze^2}{r_n}.
\end{equation}
Finally, we use the value for $r_n$ calculated in the previous section and depicted in eq. \ref{eq:5},
\begin{equation}
  E_n = - \frac{1}{8\pi\epsilon_0}\frac{Ze^2}{r_n}
   = - \frac{1}{8\pi\epsilon_0}\frac{Ze^2}{\left(\frac{a_0n^2}{Z}\right)}
   = - \frac{1}{8\pi\epsilon_0}\frac{Z^2e^2}{a_0n^2} = \hlgreen{-\frac{Z^2 E_1}{n^2}},
\end{equation}
where $E_1 = e^2/(8\pi\epsilon_0a_0) = 13.6$ eV.
\end{solution}
\end{questions}
