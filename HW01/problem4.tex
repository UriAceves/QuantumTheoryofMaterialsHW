\begin{questions}
\question{
  Atom in Octahedron
}

\begin{solution}
As we can see in the previous plots shown in Fig. \ref{5} the five $d$ orbitals have different orientations. So what we expect is:
  \begin{itemize}
    \item The orbitals $d_{xz}$, $d_{yz}$, and $d_{xy}$ will have the same energy.
    \item The orbitals $d_{3z^2-1}$ and $d_{x^2-y^2}$ will have a higher energy than the previously mentioned orbitals.
  \end{itemize}
  Naturally the energy of all of them will rise, because it will be now harder to put an electron on the d-shells, and because of the overlap of the electronic clouds.

  Following the very same idea of the overlaping clouds we can see that if we place all the ions at the same distance then the final arrangement of the d orbitals and the ions will be the same except for a rotation, but we know this does not affect the energy. Therefore we can conclude that the orbitals $d_{xz}$, $d_{yz}$, and $d_{xy}$ will have the same energy. On the other hand, we can also see that the overlap of the electronic clouds of the ions and the $d_{3z^2-1}$ and $d_{x^2-y^2}$ orbitals is greater than for the remaining 3 orbitals, thus supporting our initial guess that the orbitals $d_{3z^2-1}$ and $d_{x^2-y^2}$ will have a higher energy than the $d_{xz}$, $d_{yz}$, and $d_{xy}$ orbitals.

  Now it is a matter of calculation whether the $d_{3z^2-1}$ and $d_{x^2-y^2}$ orbitals have the same energy. It turns out they do. So at the end, we have a splitting from the degeneracy of the d states due to the field generated by a crystal field.
\end{solution}

\end{questions}
