
\begin{questions}
\question{
Reciprocal lattice of FCC
}
\begin{solution}
  Let's first define the primitive vectors for the FCC and BCC lattices. First the FCC structure
  \begin{eqnarray}
     \hat{a}_1 = \frac{a}{2}\left(\hat{y} + \hat{z}\right), \label{a1}\\
     \hat{a}_2 = \frac{a}{2}\left(\hat{z} + \hat{x}\right), \label{a2}\\
     \hat{a}_3 = \frac{a}{2}\left(\hat{x} + \hat{y}\right). \label{a3}
  \end{eqnarray}
  Then, we define the primitive vectors for the BCC lattice
  \begin{eqnarray}
     \hat{c}_1 = \frac{a}{2}\left(\hat{y} + \hat{z} - \hat{x}\right), \label{c1}\\
     \hat{c}_2 = \frac{a}{2}\left(\hat{z} + \hat{x} - \hat{y}\right), \label{c2}\\
     \hat{c}_3 = \frac{a}{2}\left(\hat{x} + \hat{y} - \hat{z}\right). \label{c3}
  \end{eqnarray}
  Here we used different names, $\hat{a}_i$, and $\hat{c}_i$ for convenience. Later we will need to be able to recognize vectors from one basis or another. Another important thing is to remember that $\hat{x}, \hat{y}, \hat{z}$ are orthonormal vectors.

  First, we will obtain the reciprocal basis for the FCC structure. For this it will be helpful to calculate the box product separately.
  \begin{equation*}
    \hat{a}_i \cdot (\hat{a}_2\times\hat{a}_3),
  \end{equation*}
  using eqs. \ref{a1}-\ref{a3}
  \begin{equation}
    \begin{aligned}[b]
      \hat{a}_1 \cdot (\hat{a}_2\times\hat{a}_3) &= \frac{a^2}{4}\hat{a}_1 \cdot \left(\left(\hat{z} + \hat{x}\right)\times\left(\hat{x} + \hat{y}\right)\right),\\
       &= \frac{a^2}{4}\hat{a}_1 \cdot \left(\hat{y} + \hat{z} - \hat{x} \right),\\
       &= \frac{a^3}{8}\left(\hat{y} + \hat{z} \right)\cdot \left(\hat{y} + \hat{z} - \hat{x} \right), \\
       &= \frac{a^3}{8}\left(\cancelto{1}{\hat{y}\cdot\hat{y}} + \cancelto{1}{\hat{z}\cdot\hat{z}} \right),\\
       &= \frac{a^3}{4}.
       \label{vol}
    \end{aligned}
  \end{equation}
  We can proceed with out calculations of the reciprocal vectors,
  \begin{equation}
    \begin{aligned}[b]
      \hat{b}_1 &= \frac{2\pi\hat{a}_2\times\hat{a}_3}{\hat{a}_1 \cdot (\hat{a}_2\times\hat{a}_3)},\\
      &= \frac{2\pi a^2/4 \left(\left(\hat{z} + \hat{x}\right)\times\left(\hat{x} + \hat{y}\right)\right)}{a^3/4},\\
      &= \hlgreen{\frac{2\pi}{a} \left(\hat{y} + \hat{z} - \hat{x} \right).}
      \label{b1}
    \end{aligned}
  \end{equation}
  Now the next one
  \begin{equation}
    \begin{aligned}[b]
      \hat{b}_2 &= \frac{2\pi\hat{a}_3\times\hat{a}_1}{\hat{a}_1 \cdot (\hat{a}_2\times\hat{a}_3)},\\
      &= \frac{2\pi a^2/4 \left(\left(\hat{x} + \hat{y}\right)\times\left(\hat{y} + \hat{z}\right)\right)}{a^3/4},\\
      &= \hlgreen{\frac{2\pi}{a} \left(\hat{z} + \hat{x} - \hat{y} \right),}
      \label{b2}
    \end{aligned}
  \end{equation}
  And the last one,
  \begin{equation}
    \begin{aligned}[b]
      \hat{b}_3 &= \frac{2\pi\hat{a}_1\times\hat{a}_2}{\hat{a}_1 \cdot (\hat{a}_2\times\hat{a}_3)},\\
      &= \frac{2\pi a^2/4 \left(\left(\hat{y} + \hat{z}\right)\times\left(\hat{z} + \hat{x}\right)\right)}{a^3/4},\\
      &= \hlgreen{\frac{2\pi}{a} \left(\hat{x} + \hat{y} - \hat{z} \right).}
      \label{b3}
    \end{aligned}
  \end{equation}
  And now we see, that the vectors defined in eqs. \ref{b1}-\ref{b3} are parallel to the primitive vectors of the BCC structure defined in eqs. \ref{c1}-\ref{c3}. \textbf{Therefore, the reciprocal lattice of a FCC one is a BCC one}. $_\blacksquare$

  As we know, if we take now the $\hat{b}$ vectors of our resulting BCC structure and calculate the resulting reciprocal ones, we will go back to the original lattice, which was a FCC lattice (the reciprocal of the reciprocal lattice is the original). \textbf{Therefore, the reciprocal of a BCC lattice is an FCC one}. $_\blacksquare$
\end{solution}
\question{Volumes}
\begin{solution}
  We now that the volume of the primitive cell is given by the box product. In eq. \ref{vol} we calculated the volume of the initial lattice. Let's now calculate the volume of the lattice formed by the $\hat{b}$ vectors.
  \begin{equation}
    \begin{aligned}[b]
      \hat{b}_1 \cdot (\hat{b}_2\times\hat{b}_3) &= \frac{(2\pi)^2}{a^2}\hat{b}_1 \cdot \left(\left(\hat{z} + \hat{x}- \hat{y}\right)\times\left(\hat{x} + \hat{y} - \hat{z}\right)\right),\\
      &= \frac{(2\pi)^2}{a^2}\hat{b}_1 \cdot \left(\hat{y} - \hat{x} + \hat{z} + \hat{x} + \hat{y} + \hat{z}\right),\\
      &= \frac{(2\pi)^2}{a^2}\hat{b}_1 \cdot \left(2\hat{y} + 2\hat{z}\right),\\
      &= \frac{(2\pi)^3}{a^3}2\left(\hat{y} + \hat{z} - \hat{x}\right) \cdot \left(\hat{y} + \hat{z}\right),\\
      &= \frac{(2\pi)^3}{a^3}2\left(1+1\right),\\
      &= \frac{(2\pi)^3}{a^3}4,\\
      &= \frac{(2\pi)^3}{a^3/4}.
      \label{vols}
    \end{aligned}
  \end{equation}
  If we remember, the volume of the direct lattice was $a^3/4$, hence, \textbf{the volume of the reciprocal cell is $v_{rec} = (2\pi)^3/v_{dir}$}. Where $v_{rec}$ is the colume of the reciprocal cell, and $v_{dir}$ the volume of the direct cell.
\end{solution}
\end{questions}

%
% \begin{center}
%   \includegraphics[width=55mm]{}
% \end{center}
%
% \captionof{figure}{}\label{new}\vspace{0.5cm}
