\begin{questions}
\question{
Fourier expansion
}
\begin{solution}
  First we know that the eigenfunctions of a Hermitian operator are orthogonal. Second, we know that $i\nabla$ is a hermitian operator (basically because the momentum is hermitian). And we now that the eigenfunctions of such an operator are plane waves
  \begin{equation}
    i\nabla e^{-iG_n\cdot r} = e^{-iG_n\cdot r}.
  \end{equation}
  Therefore we know that
  \begin{equation}
    \braket{e^{-iG_n\cdot r}|e^{-iG_m\cdot r}} = V\delta_{nm}.
  \end{equation}
  Where V is the volume over which we are integrating (since the plane waves are not normalized already).
  Hence, if we take this last equation with $m= 0$ we will find
  \begin{equation}
    \int_{\mathcal{V}} e^{-iG_n\cdot r}dr = V\delta_{n0},
  \end{equation}
  Solving the equation for delta will yield
  \begin{equation}
    \frac{1}{V}\int_{\mathcal{V}} e^{-iG_n\cdot r}dr =\delta_{n0}. _\blacksquare
  \end{equation}
  Once we have this proving the fourier expansion is easy.
  \begin{equation}
    \begin{aligned}[b]
      F(r) &= \sum_n F_n\exp{(iG_n\cdot r)},\\
      F(r)\exp{(-iG_m\cdot r)} &= \sum_n F_n\exp{(iG_n\cdot r)}\exp{(-iG_m\cdot r)},\\
      \int_{\mathcal{V}}F(r)\exp{(-iG_m\cdot r)} &= \int_{\mathcal{V}}\sum_n  F_n\exp{(iG_n\cdot r)}\exp{(-iG_m\cdot r)},      \\
      \int_{\mathcal{V}}F(r)\exp{(-iG_m\cdot r)} &= \sum_n  F_n\int_{\mathcal{V}}\exp{(iG_n\cdot r)}\exp{(-iG_m\cdot r)}, \\
      \int_{\mathcal{V}}F(r)\exp{(-iG_m\cdot r)} &= \sum_n  F_nV\delta_{mn},\\
      \int_{\mathcal{V}}F(r)\exp{(-iG_m\cdot r)} &=   F_mV.
      \label{almost}
    \end{aligned}
  \end{equation}
  Since $m$ is a mute index we can actually name it however we want, in this case $n$.   So from eq. \ref{almost} we conclude
  \begin{equation}
    F_m = \frac{1}{V}\int_{\mathcal{V}}F(r)\exp{(-iG_m\cdot r)}._\blacksquare
  \end{equation}
\end{solution}
\question{Integral over Brillouin zone}
\begin{solution}
  Let's take different cases first when $n = -m$
  \begin{equation}
    \frac{V}{(2\pi)^3}\int_{BZ} \cancelto{1}{\exp{[ik\cdot(R_{-m} + R_m)]}}dk = \frac{V}{(2\pi)^3} \int_{BZ} dk,
  \end{equation}
  But we know that the volume of the Brillouin zone is $(2\pi)^3/V$, hence
  \begin{equation}
    \frac{V}{(2\pi)^3} \int_{BZ} dk = 1.
  \end{equation}
  If $n\neq-m$ we will have
  \begin{equation}
    \frac{V}{(2\pi)^3}\int_{BZ} \exp{[ik\cdot(R_{n} + R_m)]}dk = \frac{V}{(2\pi)^3} \left.\frac{\exp{[ik\cdot(R_{n} + R_m)]}}{ik\cdot(R_{n} + R_m)}\right|_{\partial BZ},
  \end{equation}
  If we center the Brillouin zone in the origin, it can help us. Because the $k$ vectors of the Brillouin zone are periodic in such zone. So they will have to have the same values the boundaries of the Brillouin zone $\partial BZ$ but at opposite sides, so they will actually cancel each other (since we centered the Brillouin zone on the origin). Other way to see this is that we are integrating a function over its period, so the result will be zero. Therefore we conclude
  \begin{equation}
    \frac{V}{(2\pi)^3}\int_{BZ} \exp{[ik\cdot(R_{n} + R_m)]}dk = 0.
  \end{equation}
  And finally we have
  \begin{equation}
    \frac{V}{(2\pi)^3}\int_{BZ} = \delta_{n,-m}
  \end{equation}
\end{solution}

\question{Deltas 1}
\begin{solution}
  For this I will show how it's done for 1D and the extension will be easy to do. First we need to consider the lattice function
  \begin{equation}
    f(x) = \sum_{n=-\infty}^\infty \delta(x-na).
  \end{equation}
  This represents a function with period $a$, so we can also expand in fourier series the same function
  \begin{equation}
    f(x) = \sum_{h=-\infty}^\infty c_h \exp \left(i2\pi h\frac{x}{a}\right).
  \end{equation}
  We are interested on the coefficients $c_h$, so lets do that
  \begin{equation}
    \begin{aligned}
c_h &= \frac{1}{|a|} \int_{-a/2}^{a/2}f(x)\exp\left(i2\pi h\frac{x}{a}\right)dx,\\
&= \frac{1}{|a|} \int_{-a/2}^{a/2}\sum_{n=-\infty}^\infty \delta(x-na)\exp\left(i2\pi h\frac{x}{a}\right)dx,\\
&= \frac{1}{|a|} \int_{-a/2}^{a/2} \delta(x)\exp\left(i2\pi h\frac{x}{a}\right)dx,\\
\frac{1}{|a|}.
    \end{aligned}
  \end{equation}
  Therefore
  \begin{equation}
    \sum_{n=-\infty}^\infty \delta(x-na) = \frac{1}{|a|}\sum_{h=-\infty}^\infty  \exp \left(i2\pi h\frac{x}{a}\right) = \frac{1}{|a|}\sum_{h=-\infty}^\infty  \exp \left(i k_x x\right).
  \end{equation}
  If we extend this procedure to 3D we will find
  \begin{equation}
    \sum_{n=-\infty}^\infty \delta(k-G_n)  = \frac{V}{(2\pi)^3}\sum_{n=-\infty}^\infty  \exp \left(i k\cdot r_n\right).
  \end{equation}
\end{solution}
\question{Last one}
\begin{solution}
  Now lets consider the following lattice function
  \begin{equation*}
    \begin{aligned}
      &\sum_{n=-\infty}^\infty \delta(r-R_n) = \\
      &= FT^{-1} \left( \frac{1}{B^N}\frac{1}{V} \sum_{h=-\infty}^\infty \delta \left(r' - \frac{2\pi}{k}(h_1b_1+\cdots+ h_Nb_n) \right)\right),\\
      &=\frac{1}{V} \int_{-\infty}^\infty \sum_{h=-\infty}^\infty \delta \left(r - \frac{2\pi}{k}(h_1b_1+\cdots+ h_Nb_n) \right)\exp(ikr'\cdot r)dr',\\
      &=\frac{1}{V}  \sum_{h=-\infty}^\infty \exp[ i2\pi \left((h_1b_1+\cdots+ h_Nb_n)\cdot r \right],\\
      %&= \frac{1}{V}  \sum_{h=-\infty}^\infty \exp[iG_n\cdot r]. _\blacksquare
    \end{aligned}
  \end{equation*}
      \begin{equation}
        \begin{aligned}
          &= \frac{1}{V}  \sum_{h=-\infty}^\infty \exp[iG_n\cdot r]. _\blacksquare
        \end{aligned}
          \end{equation}
\end{solution}
\end{questions}

%
% \begin{center}
%   \includegraphics[width=55mm]{}
% \end{center}
%
% \captionof{figure}{}\label{new}\vspace{0.5cm}
