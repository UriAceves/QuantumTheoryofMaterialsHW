
\begin{questions}
\question{
Variational principle
}
\begin{solution}
  We will consider a Hamiltonian of the type
  \begin{equation}
    \hat{H_e} = -\frac{1}{2} \sum_i^{N_e} \nabla^2_i - \sum_i^{N_e}\sum_{I}^{N_e} \frac{Z_I}{ \vec{r}_i - \vec{R}_I } + \frac{1}{2}\sum_i^{N_e}\sum_{i\neq j}^{N_e} \frac{1}{|\vec{r}_i -\vec{r}_j|},
  \end{equation}
  if we separate the single particle and the pair functions we will obtain
  \begin{equation}
    \hat{H_e} =  \sum_i^{N_e} \hat{h}_1(\vec{x}_i) + \frac{1}{2}\sum_{i\neq j}^{N_e} \hat{h}_2(\vec{x}_i,\vec{x}_j).
  \end{equation}
  We know that
  \begin{equation}
    \braket{\Phi|\hat{H_e}|\Phi} = \sum_i^N \braket{\phi_i|\hat{h_1}|\phi_i} + \frac{1}{2} \sum_{i,j}^N \left[ \braket{\phi_i \phi_j|\hat{h_2}|\phi_i \phi_j} -  \braket{\phi_j \phi_i|\hat{h_2}|\phi_i \phi_j}\right],
  \end{equation}
  and we want to solve the following problem
  \begin{equation}
    \phi_k \rightarrow \phi_k + \delta \phi_k \quad \Rightarrow \quad \delta \braket{\Phi|\hat{H_e}|\Phi}.
  \end{equation}
  We demand the langrange multipliers that the set of orbitals $\phi_k$ remain orthogonal throughout the minimization. This is fulfilled if
  \begin{equation}
    \delta F = \delta \left[ \braket{\Phi|\hat{H_e}|\Phi} - \sum_{i,j} \lambda_{ij} ( \braket{\phi_i|\phi_j} - \delta_{ij})  \right] = 0.
  \end{equation}
  Let us see first then, how the single-body term changes
  \begin{equation}
    \delta \braket{\Phi|\sum_i^{N_e} \hat{h}_1(\vec{x}_i)|\Phi} = \braket{\delta \phi_k|\hat{h_1}|\phi_k} + \braket{\phi_k|\hat{h_1}|\delta \phi_k} = \braket{\delta \phi_k|\hat{h_1}|\phi_k} + \braket{\delta \phi_k|\hat{h_1}|\phi_k}^*.
  \end{equation}
  Now let's proceed with the second term
  \begin{equation}
    \begin{aligned}[b]
      \delta \braket{\Phi|\frac{1}{2}\sum_{i\neq j}^{N_e} \hat{h}_2(\vec{x}_i,\vec{x}_j)|\Phi} &= \delta \frac{1}{2} \sum_{i,j} \left\{ \braket{\phi_i \phi_j|\hat{h_2}|\phi_i \phi_j} -  \braket{\phi_j \phi_i|\hat{h_2}|\phi_i \phi_j}  \right\} \\
      &= \frac{1}{2} \sum_i \big\{ \braket{\phi_i \delta\phi_k | \hat{h_2}|\phi_i \phi_k} + \braket{\phi_i \phi_k | \hat{h_2}|\phi_i \delta\phi_k} \\
      & - \braket{\delta\phi_k \phi_i | \hat{h_2}|\phi_i \phi_k} - \braket{\phi_k \phi_i | \hat{h_2}|\phi_i\delta \phi_k}\big\}\\
      &+ \frac{1}{2} \sum_j \big\{ \braket{\delta\phi_k \phi_j | \hat{h_2}|\phi_k \phi_j} + \braket{\phi_k \phi_j | \hat{h_2}|\delta\phi_k \phi_j} \\
      & - \braket{\phi_j \delta\phi_k | \hat{h_2}|\phi_k \phi_j} - \braket{\phi_j \phi_k | \hat{h_2}|\delta\phi_k \phi_j}\big\}.
    \end{aligned}
  \end{equation}
  To simplify we can use the identity
  \begin{equation}
    \braket{\phi_1 \phi_2 |\hat{A}|\phi_3 \phi_4} = \braket{\phi_2 \phi_3 |\hat{A}|\phi_4 \phi_3}.
  \end{equation}
  And then, the equation simplifies to
  \begin{equation}
    \begin{aligned}[b]
      \delta \braket{\Phi|\frac{1}{2}\sum_{i\neq j}^{N_e} \hat{h}_2(\vec{x}_i,\vec{x}_j)|\Phi} &= \sum_i\big\{ \braket{\phi_i \delta\phi_k | \hat{h_2}|\phi_i \phi_k} + \braket{\phi_i \delta \phi_k | \hat{h_2}|\phi_i \phi_k}^* \\
      & - \braket{\delta\phi_k \phi_i | \hat{h_2}|\phi_i \phi_k} - \braket{\delta \phi_k \phi_i | \hat{h_2}|\phi_i \phi_k}^* \big\}.
    \end{aligned}
  \end{equation}
  Writing everything together we have
  \begin{equation}
    \begin{aligned}[b]
      \delta F &= \braket{\delta \phi_k|\hat{h_1}|\phi_k} + \braket{\delta \phi_k|\hat{h_1}|\phi_k}^* + \sum_i\big\{ \braket{\phi_i \delta\phi_k | \hat{h_2}|\phi_i \phi_k} + \braket{\phi_i \delta \phi_k | \hat{h_2}|\phi_i \phi_k}^*  - \braket{\delta\phi_k \phi_i | \hat{h_2}|\phi_i \phi_k} \\
      & - \braket{\delta \phi_k \phi_i | \hat{h_2}|\phi_i \phi_k}^* \big\} - \sum_i [\lambda_{ik} \braket{\delta \phi_k|\phi_i}^* + \lambda_{ki}\braket{\delta \phi_k|\phi_i}].
    \end{aligned}
  \end{equation}
  We are going to do the variation with respect to $\phi^*$. In terms of integrals the terms for the two-body potential look like
  \begin{equation}
    \braket{\delta\phi_k \phi_i | \hat{h}_2|\phi_i \phi_k} = \int \int \delta \phi_k^* \phi_i^* \hat{h}_2(\vec{x}_1, \vec{x}_2)\phi_i \phi_k d\vec{x}_1 d\vec{x}_2.
  \end{equation}
  As we know, if we calculate the functional derivative $\delta F / \delta \phi_k^*$ this has as effect getting rid of the term $\delta\phi_k$ in the bra and therefore the integral over its argument. This leads us to
  \begin{equation}
    \hat{h}_1 \phi_k(\vec{x}_i) + \sum_i \left\{ \int \phi_i^*(\vec{x}_2) \hat{h}_2 (\phi_i(\vec{x}_2)\phi_k(\vec{x}_1)) d\vec{x}_2 - \int \phi_i^*(\vec{x}_2) \hat{h}_2 (\phi_i(\vec{x}_1)\phi_k(\vec{x}_2)) d\vec{x}_2\right\} = \sum_i \lambda_{ki} \phi_i(\vec{x}_1).
  \end{equation}
  This last equation has the form we are looking for, so we can define an operator $\hat{F}$ on the left hand side such that
  \begin{equation}
    \hat{F}\phi_k = \sum_i \lambda{ki} \phi_i.
  \end{equation}
  There are several solutions to this equation, we can constrain ourselves to look fo the ones that satisfy
  \begin{equation}
    \lambda_{ki} = \delta_{ki}\epsilon_k,
  \end{equation}
  by doing this our equation finally changes to
  \begin{equation}
    \hat{F}\phi_k = \epsilon_k \phi_k.
  \end{equation}
  These last equations are the Hartree Fock equations.
 \end{solution}

\end{questions}

%
% \begin{center}
%   \includegraphics[width=55mm]{}
% \end{center}
%
% \captionof{figure}{}\label{new}\vspace{0.5cm}
